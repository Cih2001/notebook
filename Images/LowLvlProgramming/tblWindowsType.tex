
\begin{table}[h]
	\makegapedcells
	\centering
	\resizebox{\textwidth}{!}{%resizing the whole table
\begin{tabular}{|c|P{13cm}|}\hline
\multicolumn{1}{c}{ \bfseries Type} & \multicolumn{1}{c}{\bfseries Explanation} \\ \hline 
\textbf{BOOL(BOOLEAN)} & A Boolean variable can have the values TRUE or FALSE. Note that this is not the same as the C++ type bool, which can have the values true or false. \\ \hline
\textbf{BYTE} & An 8-bit byte. \\ \hline
\textbf{CHAR} & An 8-bit character. \\ \hline
\textbf{DWORD} & A 32-bit unsigned integer that corresponds to type unsigned long in C++. \\ \hline
\textbf{HANDLE} & A handle to an object — a handle being a 32-bit integer value that records the location of an object in memory, or 64-bit when compiling for 64-bit. \\  \hline
\textbf{HBRUSH} & A handle to a brush, a brush being used to fi ll an area with color. \\  \hline
\textbf{HCURSOR} & A handle to a cursor. \\  \hline
\textbf{HDC} & Handle to a device context — a device context being an object that enables you to draw on a window. \\  \hline
\textbf{HINSTANCE} & Handle to an instance. \\  \hline
\textbf{LPARAM} & A message parameter. \\ \hline
\textbf{LPCTSTR} & LPCWSTR if \_UNICODE is defined, otherwise LPCSTR. \\  \hline
\textbf{LPCWSTR} & A pointer to a constant null-terminated string of 16-bit characters. \\  \hline
\textbf{LPCSTR} & A pointer to a constant null-terminated string of 8-bit characters. \\  \hline
\textbf{LPHANDLE} & A pointer to a handle. \\ \hline
\textbf{LRESULT} & A signed value that results from processing a message. \\ \hline
\textbf{WORD} & A 16-bit unsigned integer, so it corresponds to type unsigned short in
C++. \\  \hline
\end{tabular}
}
\caption{Windows data types and naming}
\end{table}
